\begin{abstract}
    This paper presents a comprehensive analysis of three bottom-up subspace clustering algorithms: \textit{MAFIA}, \textit{CLIQUE}, and \textit{SUBCLU}. MAFIA extends the grid-based approach of CLIQUE by introducing adaptive grid sizes, offering improved scalability and clustering quality. SUBCLU, in contrast, utilizes a density-connectivity method that allows better identification of arbitrarily shaped clusters, overcoming some of the limitations inherent in grid-based algorithms. The paper explores the relationships and distinctions between these algorithms, evaluating their performance in terms of scalability of dataset size, data- and cluster-dimensionality, as well as their clustering quality. Through a series of experiments with both synthetic and real-world datasets, we demonstrate the strengths and limitations of each approach. The findings reveal that while MAFIA is the most scalable, SUBCLU excels at detecting clusters with irregular shapes. However, all algorithms require proper parameter tuning for optimal performance.
    
    \keywords{High-Dimensional Subspace Clustering \and Grid-Based- and Density-Connectivity approach \and Comparative Study.}
\end{abstract}