\section{Discussion}
CLIQUE, MAFIA, and SUBCLU are three prominent subspace clustering algorithms, each offering unique strengths and limitations. CLIQUE, as one of the earliest subspace clustering approaches, uses a grid-based method where the data space is partitioned into equi-sized cells, and dense units are identified based on a density threshold. The advantage of CLIQUE is its simplicity and efficiency in navigating through subspaces using an Apriori-like strategy. However, its reliance on fixed grid boundaries can lead to issues when clusters do not align with these grids, resulting in missed or fragmented clusters. Moreover, the algorithm is sensitive to grid size and density thresholds, which may not be optimal for all datasets.

MAFIA builds upon CLIQUE by introducing adaptive grid sizes, allowing it to dynamically adjust grid boundaries based on data density. This modification enables MAFIA to detect clusters more precisely, providing better resolution than CLIQUE, especially in complex datasets. Additionally, experiments indicate that MAFIA is the most scalable algorithm when it comes to handling large data sizes. However, this scalability comes at a cost; MAFIA is highly sensitive to input parameters, such as grid size and density thresholds, and requires careful tuning to achieve optimal results. While it offers more precise boundary detection than CLIQUE, its dependence on parameters can lead to inconsistent performance across different datasets.

SUBCLU, on the other hand, departs from the grid-based approach and utilizes a density-connectivity principle similar to DBSCAN. This enables it to detect clusters of arbitrary shapes, which is a significant advantage when clusters are irregular or not aligned with the axes. This was demonstrated in experiments using real datasets, where SUBCLU outperformed grid-based methods in identifying clusters with complex boundaries. However, SUBCLU’s approach, while flexible, may not be as efficient as MAFIA in terms of data size scalability, particularly for very large datasets.

A critical observation across these algorithms is that their performance can vary significantly depending on the data characteristics. Experiments using synthetic data show that clustering quality is highly use-case dependent, and allowing or disallowing lower-dimensional clusters can affect outcomes. Furthermore, while these synthetic experiments highlight each algorithm’s strengths, they also reveal their limitations. It is possible to generate datasets where none of these methods can identify the correct clusters, showing that subspace clustering remains a challenging problem.

All three algorithms struggle with closely clustered data points, often requiring fine-tuning of parameters such as grid size in CLIQUE and MAFIA, or \( \epsilon \) and \( m \) in SUBCLU, to effectively separate clusters. This need for parameter optimization underscores the complexity of subspace clustering and the necessity for tailored solutions depending on the specific data distribution and clustering objectives.