\section{Discussion}
Experiments have shown that MAFIA is the most scalable algorithm for large datasets, primarily due to its adaptive grid approach, which optimizes both computational efficiency and clustering quality. CLIQUE, while also scalable, struggles with clustering quality, particularly when the grid size is not well-tuned, limiting its ability to detect clusters in more complex data distributions. SUBCLU, on the other hand, excels at detecting arbitrarily shaped clusters through its density-connected approach, though this comes at a higher computational cost compared to MAFIA and CLIQUE. All three algorithms require careful parameter tuning for optimal performance, and they exhibit sensitivity to closely clustered data. The datasets used in the experiments can easily favor one algorithm over another depending on their structure.

One of the key differences between MAFIA and the other two algorithms is that MAFIA relies on both \textit{cluster parameters} and \textit{algorithm parameters}, whereas CLIQUE and SUBCLU use only algorithm parameters \cite[p.~342]{sim-2012}. Cluster parameters define the characteristics of the clusters themselves, while algorithm parameters guide the clustering process. This additional flexibility allows MAFIA to better adapt to the underlying data structure during clustering. However, this also complicates the tuning process as it introduces more parameters that must be optimized. Beyond parameters such as $\alpha$ and $\beta$, MAFIA also depends on the minimum number of bins (denoted $n$) and the maximum number of windows (denoted $M$). $n$ controls the initial number of bins generated in the 1-dimensional space, while $M$ limits the number of windows that can be generated to avoid exponential growth as the dimensionality increases. The choice of $n$ and $M$ is crucial for MAFIA's performance, but these parameters were not thoroughly discussed in the original paper.

A shared challenge across all three algorithms is their reliance on global parameters for determining which regions are considered dense. As noted in \cite[p.1:16]{kriegel-2009}, using a global density threshold can lead to bias towards certain dimensionalities, where a stricter threshold may miss high-dimensional clusters while a more relaxed one may generate too many low-dimensional clusters. MAFIA attempts to address this by adjusting the grid size adaptively, but the global parameter problem persists across all three methods.

The decision to use only the best parameters for the algorithms is based on the discussion in \cite[p.~352]{sim-2012}, which highlights that such parameters are often "non-meaningful and non-intuitive." The authors recommend testing a range of parameters, iteratively adjusting them until an appropriate number of clusters is found. This same approach was applied in this study. However, it is possible that better parameter combinations exist, which were not identified in this process, and could further improve the performance of the algorithms. Nevertheless, we prefer algorithms which are not sensitive to different parameter values.

The trade-offs between scalability, clustering quality, and parameter sensitivity make it crucial to consider the specific requirements of each use case when choosing between CLIQUE, MAFIA, and SUBCLU. For large datasets where computational efficiency is key, MAFIA's adaptive grid approach offers significant advantages. However, for datasets containing complex or irregularly shaped clusters, SUBCLU's density-based method proves superior, albeit with a higher computational cost. Also, as noted by Kriegel et al. \cite[p.1:50]{kriegel-2009}, ''the trade-off between efficiency and effectiveness is tolerable will depend on the application''.