\section{Conclusion}
In conclusion, this study highlighted the strengths and limitations of three subspace clustering algorithms, with a particular focus on MAFIA. MAFIA's adaptive grid approach has proven to be the most scalable, making it ideal for large datasets where computational efficiency is paramount. Its flexibility in adapting to the data structure allows for better performance, but this comes with the challenge of more complex parameter tuning. CLIQUE, while also scalable, suffers from reduced clustering quality in more complex data distributions due to its reliance on fixed grid sizes. SUBCLU, with its density-connected method, excels at detecting arbitrarily shaped clusters, but its higher computational cost limits its scalability. MAFIA's ability to balance scalability and clustering quality, particularly in large-scale data sets, ,makes its a powerful tool in subspace clustering. However, future work could further refine MAFIA's parameter sensitivity.