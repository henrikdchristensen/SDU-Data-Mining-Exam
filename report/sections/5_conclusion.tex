\section{Conclusion}
The analysis of CLIQUE, MAFIA, and SUBCLU highlights the strengths and limitations of each algorithm in subspace clustering. CLIQUE offers a straightforward, grid-based approach suitable for detecting axis-aligned clusters, but its reliance on fixed grids and sensitivity to parameters can limit performance. MAFIA improves upon CLIQUE by adapting grid sizes, providing better resolution and scalability for large datasets, though it still remains somewhat dependent on parameter tuning. SUBCLU, using a density-based method, excels in identifying arbitrarily shaped clusters, especially in real-world data, but it may not scale as efficiently as CLIQUE and MAFIA.

Overall, the choice of algorithm depends on the use case, data size, and the shape of clusters. The experiments demonstrated that these algorithms perform well under controlled conditions, but in highly complex data sets, none of the algorithms detects clusters accurately. Hence, while each method has its advantages, they all require careful adjustment of parameters and may need further refinement for optimal performance in diverse datasets.