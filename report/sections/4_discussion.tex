\section{Discussion}
Experiments have shown that MAFIA is the most scalable algorithm for large datasets due to its adaptive grid approach, which enhances both computational efficiency and clustering quality. While CLIQUE is also scalable, it struggles with clustering quality, especially when the grid size is not well-tuned. SUBCLU, on the other hand, excels in identifying arbitrarily shaped clusters through its density-connected approach, but this comes at a higher computational cost compared to MAFIA and CLIQUE.

Both CLIQUE and MAFIA have shown to scale linearly with the dataset size. In contrast, SUBCLU exhibits a quadratic growth, as it relies on partial range queries \cite{subclu}. In terms of cluster dimensionality, CLIQUE exhibits a quadratic growth rate because the number of generated CDUs increases exponentially. In contrast, MAFIA generally shows a linear growth rate as cluster dimensionality increases. However, MAFIA's runtime rapidly increased once the number of dimensions exceeded 15, likely due to system limitations rather than the algorithm itself, as a linear growth rate is reported in \cite{mafia}. Further research is needed to investigate this behavior. Due to high memory consumption, SUBCLU could not be tested on this dataset, but in \cite{subclu} it is noted that it has at least a quadratic growth rate with respect to cluster dimensionality. Also the number of dimensions in the dataset does not affect growth rate of CLIQUE and MAFIA.

All three algorithms require careful parameter tuning for optimal performance, as observed in the clustering quality experiments. However, a key distinction between MAFIA and the other two algorithms is that MAFIA depends on both \textit{cluster parameters} (i.e. $\alpha$) and \textit{algorithm parameters}, whereas CLIQUE and SUBCLU solely on algorithm parameters, such as the global density threshold \cite[p.~342]{sim-2012}. By using cluster parameters, MAFIA can set thresholds for how strong a cluster must be to be considered. However, this also complicates the tuning process by introducing more parameters to optimize. In addition to parameters like $\alpha$ and $\beta$, MAFIA also depends on the minimum number of bins ($n$) created at the start and the maximum number of windows ($M$) generated to prevent exponential growth as dimensionality increases. Both $n$ and $M$ were not extensively discussed in the original paper. The clustering accuracy experiments clearly show that the performance of the algorithms is highly dependent on the structure of the data. It is also possible that better parameter combinations exist for an algorithm, which were not identified in these tests, and could further improve its performance. It should be noted, however, that algorithms less sensitive to parameter variations are generally preferred.

The trade-offs between scalability, clustering quality, and parameter sensitivity make it crucial to consider the specific needs of each application when choosing between CLIQUE, MAFIA, and SUBCLU. For large datasets where computational efficiency is a priority, MAFIA's adaptive grid approach offers significant advantages. However, for datasets with complex or irregularly shaped clusters, SUBCLU's density-based method is superior, albeit at a higher computational cost. As Kriegel et al. note \cite[p.1:50]{kriegel-2009}, the trade-off between efficiency and effectiveness depends on the application.