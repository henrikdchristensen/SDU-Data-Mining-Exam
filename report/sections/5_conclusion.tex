\section{Conclusion}
In conclusion, this study highlighted the strengths and limitations of three subspace clustering algorithms. MAFIA's adaptive grid approach has proven to be the most scalable, making it ideal for large datasets where computational efficiency is required. Its flexibility in adapting to the data structure allows for better performance than CLIQUE, but this comes with the challenge of more complex parameter tuning. CLIQUE, while also scalable, suffers from reduced clustering quality in more complex data structures due to its reliance on fixed grid sizes. SUBCLU, with its density-connected method, excels at detecting arbitrarily shaped clusters, but its higher computational cost limits its scalability. MAFIA's ability to balance scalability and clustering quality, particularly in large-scale data sets, makes its a powerful algorithm in subspace clustering.

\paragraph{Future Work.}
In this study, internal and external evaluation metrics were not directly used to compare the clustering results of the three algorithms. The lack of external metrics is primarily due to the limitations of those available in ELKI, which are not well-suited for subspace clustering \cite{e4sc}. Therefore, it would be valuable to explore the subspace clustering evaluation metric \textit{E4SC}, as proposed by \cite{e4sc}. Furthermore, a deeper investigation into MAFIA's sensitivity to parameter settings, as well as an exploration of SUBCLU's scalability with respect to cluster and data dimensionality, would be valuable.